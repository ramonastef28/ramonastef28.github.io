% Cover letter using letter.sty
\documentclass{letter} % Uses 10pt
%Use \documentstyle[newcent]{letter} for New Century Schoolbook postscript font
% the following commands control the margins:
\topmargin=-1in    % Make letterhead start about 1 inch from top of page 
\textheight=8in  % text height can be bigger for a longer letter
\oddsidemargin=0pt % leftmargin is 1 inch
\textwidth=6.5in   % textwidth of 6.5in leaves 1 inch for right margin

\newcommand{\Deg}{$^{\circ}$}

\begin{document}

\signature{E. Ramona Stefanescu}           % name for signature 
\longindentation=0pt                       % needed to get closing flush left
\let\raggedleft\raggedright                % needed to get date flush left
 
\begin{letter}{\\
Author response to the queries -- RSPA20110711 }


\begin{flushleft}
{\large\bf E. R. Stefanescu - corresponding author}
\end{flushleft}
\medskip\hrule height 1pt
%\begin{flushright}
%\hfill 318 Jarvis Hall, Buffalo, NY 14260 \\
%\hfill (716) 238-1179 or (716) 645-1420 
%\end{flushright} 
%\vfill % forces letterhead to top of page

 
%\opening{To whom it may concern:} 
 \begin{itemize}
 \item Q1 - OK
 \item Q2 -  Department of Civil Engineering, Universidad de Nari\~{n}o, Nari\~{n}o, Colombia 
 \item Q3 - Please replace the sentences starting line 311 with: \textit{For sampling the input parameter
space, a Latin Hypercube Sampling (LHS) was implemented. LHS is
commonly used in computer sampling experiments McKay et al., Sacks et al., mainly because
it is computationally cheap to generate and can cope with many input
variables.} 
\item Q4 -OK
\item Q5 - TOPSAR (Topographic Synthetic Aperture Radar)
\item Q6 - Please use: \textit{For any point in the domain, it can now be exercised like the simulator to get
potential flows and hence exceedance probabilities.}
\item Q7 - OK
\item Q8 - OK
\item Q9 - Ehlschlaeger, Ch.R., and  M.F. Goodchild, 1994.
\textit{Uncertainty in spatial data: Defining, visualizing and 
managing data errors}. Proceedings of GIS/LIS, Phoenix,
Arizona. pp. 246-53. \\

Stefanescu, E.R. and Bursik, M.I. and Cordoba, G. and Patra, A. , Pieri, D.C. and  M.F. Sheridan, 2010. \textit{Impact of DEM uncertainty on TITAN2D flow model output, Galeras Volcano, Colombia}. Proceedings of International Congress on Environmental Modelling and Software. International Environmental Modelling and Software Society (iEMSs), Ottawa, Canada. \\

Stefanescu, E.R., Bursik, M.I., Dalbey, K., Jones, M.D., Patra, A.K. and E.B. Pitman, 2010. \textit{DEM uncertainty and hazard analysis using a geophysical flow model}. Proceedings of International Congress on Environmental Modelling and Software. International Environmental Modelling and Software Society (iEMSs), Ottawa, Canada. \\

\item Q10 - Ehlschlaeger, Ch.R. and A. Shortridge, 1996.
\textit{Modeling elevation uncertainty in geographical analysis}
Proc. Spatial Data Handling '96, Delft, The Netherlands, v. 2 (1996), pp. 9B.15�9B.2 \\

Heuvelink, G.B.M., Burrough, P.A., and A. Stein, 1989. \textit{Propagation of errors in spatial modelling 
with GIS}. Int. J. Geogr.Inf. Syst. 3, 303�32 

\item Q11 - M. Goldstein, 1995. \textit{Bayes linear methods I - Adjusting beliefs: concepts and properties}. Technical Report 1995/1, Department of Mathematical Sciences, University of Durham. \\

Sheridan, M.F., Patra, A.K., Dalbey, K. and B., Hubbard, 2010. \textit{Probabilistic digital hazard maps for avalanches and massive pyroclastic flows using TITAN2D}. Geological Society of America Special Papers, 464, p. 281-291\\

Weng, Q., 2002. Quantifying uncertainty of digital elevation models
derived from topographic maps, Advances in Spatial Data
Handling (D. Richardson and P. van Oosterom, editors),
Springer-Verlag, New York, New York, p. 403�418

\item Q12 - OK

\item Q13 - Stefanescu, E.R., Bursik, M. and A.K., Patra, 2012. \textit{Effect of digital elevation model on Mohr-Coulomb geophysical flow model output}, Natural Hazards, doi:10.1007/s11069-012-0103-y

\item Q14 - Please remove

\item SQ1 - OK \\

\item Line 310: Change to \textit{One working hypothesis}
\item Line 393: \textit{... use of a hierarchical emulator}.
 \item Line 579:  Change to \textit{30\Deg and 35\Deg}.
 \item Line 680: Change to \textit{TITAN2D}.
 \item Line 772: Change from Mountain Mountain to \textit{Mammoth Mountain}.
 \item Line 780: Change to \textit{In previous work ... footprint}
 \item Line 1012: Change to reference to: \\
 Hildreth, W. 2004, \textit{Volcanological perspectives on Long Valley, Mammoth Mountain, and Mono Craters: Several contiguous but discrete systems}, J. Volcanol. Geotherm. Res., 136,169 � 198.
 \end{itemize}

%\encl{}  				% Enclosures

\end{letter}
 

\end{document}






