%\documentclass[a4paper]{article}
\documentclass{article}
%\documentclass[journal]{IEEEtran}
%\documentclass{report}
%\documentclass{ActaOulu}

\usepackage{graphicx}
\usepackage{multirow}
\usepackage{authblk}
\usepackage{float}
\usepackage{rotating}
\usepackage{url}
\usepackage{lscape}
\usepackage{longtable}
\usepackage{subfig}
\usepackage{natbib}
\usepackage{lineno}
\usepackage{amsmath}
\usepackage{epsfig}
\usepackage{caption}
\usepackage{latexsym}
\usepackage[a4paper]{geometry}
%\setlength{\captionmargin}{20pt}
%\usepackage{graphicx}
%\usepackage[all,knot]{xy}
%\xyoption{arc}
%\usepackage{url}
%\usepackage{multimedia}
%\usepackage{hyperref}
\linespread{1.6}
%\linespread{1.2}

\begin{document}

\title{Numerical implementation of a sparse grid for a hyperbolic geophysical mass flow model}
\author[1]{ E. R. Stefanescu }
\author[2]{K.R. Dalbey}
\author[1]{A.K. Patra}
\author[3]{M. Bursik}
\affil[1]{Department of Mechanical and Aerospace Engineering, University at Buffalo}
\affil[2]{Sandia National Labs, Albuquerque}
\affil[3]{Department of Geology, University at Buffalo}


%\date{\today}


\maketitle

\begin{abstract}
To come soon...
\end{abstract}


%\chapter{First Chapter}

\section{Introduction}
This paper describes the development of sparse grid based methodology for the propagation of input 
parameter uncertainties through a shallow-water like hyperbolic system model of 
geophysical mass flows. We apply the sparse grid based 
methodology first to the straightforward analysis of  UQ in the output of 
the model and then to the more complex construction of hazard maps based on the 
outcomes of ensembles of model runs. The results are interesting in that they show that the sparse grid 
method is quite successful in the first type of analysis where the uncertainty in a number of flow outcomes are successfully 
captured but the results from the hazard map construction that requires a more extensive exploration of the 
full parameter space to construct a response surface is not as successful.

\subsection{Background} 
\subsubsection{Need} In geophysical mass flow problems (e.g. landslides, volcanic debris flows called lahars), many flow characteristics like
material properties, the size and location of failing mass, the terrain over which the 
flow occurs at every point in the domain are difficult if not impossible to characterize. There is also much that is unknown about the
exact physics. Thus, useful outputs from the model equations governing mass flow must account for 
the variable or poorly characterized aspects of the model as well as the uncertainty of the 
parameters used in the equations (keithuncert). 
Most of the numerical models of real-world systems are characterized by large number of input
variables. In this case, to obtain a reliable numerical 
solution, one has to include uncertainty quantification due to the uncertainty in the input data. (babuska). Furthermore, 
in many use situations of geophysical mass flows, one is concerned with flow over an entire jurisdiction (hazard analysis). 
This introduces additional complexity in the propagation of the uncertainty.

\subsubsection{Uncertainty Quantification Methods} Quantification of uncertainty means to be in some way able to attach a measure
to something which may be poorly defined or vague \citep{Matthies2008} -- in this case the outcomes of models 
with uncertain inputs. 
To quantify the uncertainties in the result of a simulation, one must understand 
both the sources of these uncertainties, and how uncertainties propagate through the simulation. 
%The recent literature on such uncertainty propagation is extensive \cite{revpap1,revpap2}.
% is there has been intensively studied the impact of errors, or uncertainty ,
%in ``data" such as parameter values, initial and boundary conditions. It came an important issue
%due to mathematical models that serve as simplified and reduced representations of the true physics.
Uncertainty quantification thus involves two steps: determination of the uncertainty sources, and 
analysis of their propagation through the simulation. In the past few years there have been 
extensive studies in the propagation of uncertainty. The common stochastic methods can be
categorized in sampling and non-sampling methods (cfdsppitt). Widely used sampling methods
are: Monte-Carlo(MC), quasi-Monte Carlo (QMC), Latin Hypercube sampling (LHS), while as 
non-sampling methods intensively used are:  Polynomial Chaos Expansion (PCE), Stochastic Galerkin (SG),
Non-Intrusive Spectral Projection (NISP), Polynomial Chaos Quadrature (PCQ) etc. 

MC methods have been applied to many applications and their implementation is straightforward.
Although the convergence rate is relatively slow, it is independent of the dimensionality of the random space and
independent of the number of random variables used to characterized the random inputs (xiuhesthaven). 
To accelerate convergence in MC, several techniques have been developed as LHS and QMC.
The PCE method uses the Wiener-Askey scheme, in which Hermite, Legendre, Laguerre, Jacobi, and generalized Laguerre
orthogonal polynomials are used for modeling the effect of uncertain variable described by normal, uniform, exponential, beta, 
and gamma distributions, respectively (eldredSandia and Xiu). These orthogonal polynomial selections are optimal for these
distribution type since the inner product weighting function and its corresponding support range correspond to the probability density 
functions for these continuous distributions. 
PCQ is a based on the PCE in general and on SG in particular. The main idea is that in addition to 
calculating the coefficients in the expansion of chaos polynomials, PCQ achieves a more accurate computation of statistical 
moments when the governing system of equations is non-polynomial, by calculating the moments directly rather than
through the intermediate coefficients.
These concepts has been used successfully in many engineering approaches over the past two decades. 
%In most of the cases it involves generation of samples of input parameter followed by the application of the
%simulation model to generate the corresponding set of stochastic responses. 
All methods have positive and negative features, and no single technique is optimum for all situations.

Recently, a stochastic collocation scheme has been introduced in which simulations are performed
at specific collocation points in the stochastic space (citation). Stochastic collocation methods for 
sPDEs have been first proposed in [XiuHesthaven and Babuska]by using numerical quadrature 
for the approximate evaluation of the stochastic integrals. 
These technique combine the exponential convergence rate of the PCE scheme with the
decoupled nature of MC techniques (sandiego). The selection of computational nodes is 
the key ingredient in all stochastic colocation methods. Choices of collocation points include
tensor product of zeros of orthogonal polynomials (citation), sparse grid (SP) methods (citation),
and probabilistic collocation(cfdsppitt).


This paper presents an approach to characterize the effect of the input uncertainty on the output 
of a geophysical mass flow model. We use the concept of stochastic collocation to account for 
uncertainties in a non-linear hyperbolic system of PDEs used in modeling geophysical mass 
flow. Neglecting the model uncertainty we focus on the input parameter uncertainty, modeled as 
random variables. Geological flow models are dealing with high dimensional input. There are large 
uncertainties associated with the construction of the DEMs and many other parameters that characterize 
the flow.

We first discuss sparse grid algorithms as an improvement of the MC, LHS and PCQ methods. 
The sparse grid scheme are chosen based on their efficiency which in most of the cases
results in a number reduction of collocation points needed to obtain a given level of approximation. 
Further, a stochastic space is defined for each problem depending on the uncertainty in
parameters. Computational points are identified in this stochastic space and simulations 
of the geophysical flow model are performed. Our goal is to compute the statistics of the 
output in form of mass velocity, maximum height of the pile over time, momentum etc and
compare the convergence for the above mentioned methods. 
 

Sparse grid schemes have been successfully implemented for parabolic and elliptic sPDEs.
(Borziaustria) proposed a framework combining space-time multigrid methods with sparse grid
collocation techniques to solve a nonlinear parabolic system with random coefficients, while
(Babuskanobile) proposed a stochastic collocation method to solve elliptic partial differential
equations with random coefficients and forcing terms. In the paper of (cfdspitt) a numerical 
simulations is done for an incompressible and slightly compressible single and the two-phase 
flow in porous media. The stochastic domain is represented using collocation at the zeros of 
tensor product Hermite polynomials. It was observed that the stochastic collocation converges
much faster (to the mean and variance) than the standard MC approach with a significantly 
reduced number of simulations. In the paper (Melosh07) a comparison of the accuracy and 
efficiency of the sparse grid collocation approach to MC and GPCE is performed, when used 
to solve a natural convection problem.

\subsection{Hazard Analysis} Determining the risk associated with geophysical flows has been a effort of a large scientific
community (citation). In areas of flood hazards the delineation of  regions into zones of high and low risk
is usually based on the probability of a critical threshold for some parameter associated with 
the hazard map being exceeded.
In the paper, we also propose to investigate the use of sparse grid design (SPD) for hazard 
map construction. Sparse grids provide sampling points which avoid the curse of dimensionality.
A Bayes Linear Model (BLM) is used to fit the data and construct a hazard map. 


The paper is organized as follows: we first give a brief description of the non-linear hyperbolic systems
used in modeling geophysical mass flows. Then we proceed to discuss sparse grid schemes 
in section 3 and details of numerical implementation in section 4. Finally discussions and conclusions
are presented in section 5.


\end{document}