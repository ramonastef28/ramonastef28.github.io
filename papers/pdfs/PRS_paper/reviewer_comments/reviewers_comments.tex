\documentclass[12pt]{article}
\usepackage{graphicx}
\usepackage{multirow}
\usepackage{authblk}
\usepackage{float}
\usepackage{rotating}
\usepackage{url}
\usepackage{lscape}
\usepackage{longtable}
\usepackage{subfig}
\usepackage{natbib}
%\usepackage{subfigure}
\linespread{1.0}

%\newcommand{\Pic}[2][0.85]{\begin{center}\includegraphics[width=0.8\textwidth,height=#1\textheight,keepaspectratio]{#2}
 %\end{center} }



\title{ Response to reviewers' comments}
\author{}
\date{\today}
\begin{document}
\maketitle

We would like to thank the two reviewers for their valuable comments and constructive suggestions. We have made every effort to address your 
concerns in the revised article. \\


\textbf{Reviewer 1} \\

Changes were made in the paper to address all the points. However, the discussion on  LHS was not extended as desired by the reviewer, mainly because of the large
literature available on the design and analysis of computer experiments. A recent paper entitled `` Hazard risk maps using additive ensembles of local emulators of
complex computer model of physical phenomena'', by Dalbey, et al. describes the LHS implementation for 256 sample points for Montserrat Volcano.
In our case, we have chosen a sample design (LHS) to capture as many events and employed an importance sampling for the log(volume) such that we have an affordable computational cost for all events.\\

\textbf{Reviewer 2}\\
All the comments were addressed in the revised paper by changing wording.  No change in the science was required. 


\end{document}
