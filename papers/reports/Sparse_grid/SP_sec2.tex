%\documentclass[a4paper]{article}
\documentclass{article}
%\documentclass[journal]{IEEEtran}
%\documentclass{report}
%\documentclass{ActaOulu}

\usepackage{graphicx}
\usepackage{multirow}
\usepackage{authblk}
\usepackage{float}
\usepackage{rotating}
\usepackage{url}
\usepackage{lscape}
\usepackage{longtable}
\usepackage{subfig}
\usepackage{natbib}
\usepackage{lineno}
\usepackage{amsmath}
\usepackage{epsfig}
\usepackage{caption}
\usepackage{latexsym}
\usepackage[a4paper]{geometry}
%\setlength{\captionmargin}{20pt}
%\usepackage{graphicx}
%\usepackage[all,knot]{xy}
%\xyoption{arc}
%\usepackage{url}
%\usepackage{multimedia}
%\usepackage{hyperref}
\linespread{1.6}
%\linespread{1.2}

\begin{document}
\section{Basic model}
\subsection{Governing equations}
The geophysical mass flow code - TITAN2D combines numerical simulations of the flow with
digital data of the natural terrain. It is based on a depth-averaged
model for an incompressible granular material, governed by
Coulomb-type friction interactions \citep{Savage1989}.  The
governing equations are obtained by applying conservation laws to the
incompressible continuum, providing appropriate constitutive modeling
assumtions, and then taking advantage of the shallowness of the flows
(flows are much longer and wider than they are deep) to obtain simpler
depth-averaged representations \citep{Patra2005}. The motion of the
material is considered to be gravitationally driven and resisted by
both internal and bed friction forces. The stress boundary conditions
are: no stress at the upper free-surface and a Coulomb-like friction law
imposed at the interface between the material and the basal
surface. The resulting hyperbolic system of equations is solved using a
finite-volume scheme -- with a second-order -- Godunov solver. Even if many
real geophysical flows such as debris flow are fluidized, in this
study we deal only with granular material that has not been fluidized.

\begin{align} \nonumber
&\frac{\partial{h}}{\partial{t}} + \frac{\partial{(V_x \cdot h)}}{\partial{x}} + \frac{\partial{(V_y \cdot h)}}{\partial{y}} &= 0 \\ 
&\frac{\partial{hV_x}}{\partial{t}} + \frac{\partial{(V_x \cdot hV_x + 0.5 k_{ap}g_zh^2)}}{\partial{x}} + \frac{\partial{(V_y \cdot hV_x)}}{\partial{y}} &={S_x} \\  \nonumber
&\frac{\partial{hV_y}}{\partial{t}} + \frac{\partial{(V_x \cdot hV_y)}}{\partial{x}} + \frac{\partial{(V_y \cdot hV_y + 0.5 k_{ap}g_zh^2)}}{\partial{y}} &= S_y \\ \nonumber
\label{eqn1}
\end{align}
where 
\begin{align} \nonumber
&k_{ap} = 2\frac{1\mp \sqrt{1-cos^2(\phi_{int})(1+\tan^2(\phi_{bed}))}}{\cos^2} -  1  \\ \nonumber
\end{align}
The source terms are defined as:
\begin{align} 
&S_x = g_xh -\frac{V_x}{\sqrt{V_x^2+V_y^2}}\max{\left(g_z + \frac{V_x^2}{r_x}, 0 \right)} h \tan((\phi_{bed}) - 
-  sgn \left( \frac{\partial{V_x}}{\partial{y}} \right )h k_{ap} \frac{\partial{(g_xh)}}{\partial{y}} \sin((\phi_{int}) \\
&S_y = g_yh -\frac{V_y}{\sqrt{V_x^2+V_y^2}}\max{\left(g_z + \frac{V_y^2}{r_y}, 0 \right)} h \tan((\phi_{bed}) -  \nonumber
-  sgn \left( \frac{\partial{V_y}}{\partial{x}} \right )h k_{ap} \frac{\partial{(g_yh)}}{\partial{x}} \sin((\phi_{int}) \\  \nonumber
\end{align}

where, $h$ is the height of the flow, $hV_x, hV_y$ are moments in $x$ and $y$ directions, $g = \{g_x,g_y,g_z\}$ are
components of the gravity along the three axes. $\phi_{int}$ and $\phi_{bed}$ are the internal and basal friction angles,
respectively, and $r_x$ and $r_y$ are the radii of curvature of terrain in the $x$ and $y$ directions.

\subsection{Parameters uncertainty}
TITAN2D was developed for modeling dry geophysical granular flows,
such as debris avalanches and block and ash flows.  Given a digital
elevation map specifying the topography of a volcano and the values of
input parameters, including the initial volume of erupted material and
the friction angles, TITAN2D calculates the flow depth and velocity at
any location throughout the duration of an event. There is uncertainty 
in all of the input parameters. It was found by (Keith) that the flow is fairly
insensitive to internal friction angle, but bed friction angle and the size of the
initial flowing mass are important. The degree of sensitivity to initial location
highly depends on the local terrain features in the neighborhood of the
starting location. The combination of uncertainty and sensitivity to these
inputs dictates which ones need to be represented as random variables.

\end{document}
